\addcontentsline{toc}{section}{Procedures}
\section*{Procedure}\label{sec:procedure}

%TOPIC: Attributional LCA
In the overall, this research exercise remained along the character of an attributional LCA.
In difference from a consequential practice, this inquiry did not accounted for dynamic changes outside of systems boundaries in consequences of variations at the inputs.
Consequential descriptions as economic relationships, marginal production cost or the elasticity of supply and demands were not considered.
A confining, material-balanced approach was selected.\\

%TOPIC: About Mass Balance
%Describe Plastic Losses and plastic-to-plastic yield

%TOPIC: About Open LCA

%TOPIC: TECHNICALLY NEITHER OF THE TECHNIQUES ARE THE SAME
%Not only is there a difference in the recicling and reusage understanding...
%But even from the materials that are being used... After the sorting of the material... THE TREATMENT OF MONO FLWO AND PLY FLOWS IS DIFERENT FROM EACH TECHNNOLY
%Now, because the comparison between these two methodologies comes as farfeched... It is better to look at a bigger picture... The one that comes from the ultimate usage of those carbon molecules.


%%%%%%%%%%%%%%%%%%%%%%%%%%%%%%%%%%%%%%%%%%%%%%%%%%%%%%%%%%%%%%%%%%%%
\todo[inline]{\fontfamily{cmss}\selectfont
{Here comes a description of the methodology.
Not about the LCA itself but specificities that belong to our own exercise.
That is, for example the usage of Open LCA as a tool of choice, the databases we are pulling data from, assumptions and mass balance considerations.
}}
%%%%%%%%%%%%%%%%%%%%%%%%%%%%%%%%%%%%%%%%%%%%%%%%%%%%%%%%%%%%%%%%%%%%
