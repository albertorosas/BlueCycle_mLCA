\addcontentsline{toc}{section}{Procedures}
\section*{Procedure}\label{sec:procedure}

%TOPIC: Attributional LCA ✔
Overall, this research exercise remained along the character of an attributional LCA.
In difference from a consequential practice, this inquiry did not account for dynamic changes outside of systems boundaries in consequences of input variations.
Consequential descriptions as economic relationships, marginal production cost or the elasticity of supply and demands were not considered.
A confining, material-balanced approach was selected.

%TOPIC: About Mass Balance ✔
1,000 kg of DKR 350 material was modelled as an input across all the different product systems.
Both single and multi-loop calculations began with the acquisition of presorted recyclable material.
Nevertheless, the respecting mass balance across contrasted technologies became singular to their specific conditions.
This feature was paramount in the modelling as every subsequent unit process utilised only the output mass of its predecessor.

%TOPIC: The Databases used. ✔
The usage of specific inventory databases was also an enabling and limiting factor.
Three datasets were convened to realise this assessment: Ecoinvent v3.8, Product Environmental Footprints (PEF) and European Reference Life Cycle Database from the Joint Research Center v3.2.
These provided a comprehensive source of unit processes and elementary flow inventory but also confined the modelling exercise to how the author contemplated their boundaries.
At times, these may have overlapped and demanded a detailed verification to avoid double counting.
The latest version of OpenLCA software was used to fastly and reliably calculate the impacts across mentioned inventories.


%TOPIC: TECHNICALLY NEITHER OF THE TECHNIQUES ARE THE SAME
    %PENDING: FULL PARAGRAPH
    %Not only is there a difference in the recicling and reusage understanding...
    %But even from the materials that are being used... After the sorting of the material... THE TREATMENT OF MONO FLWO AND PLY FLOWS IS DIFERENT FROM EACH TECHNNOLY
    %Now, because the comparison between these two methodologies comes as farfeched... It is better to look at a bigger picture... The one that comes from the ultimate usage of those carbon molecules.
