\chapter{Inventory Analysis}\label{ch:inventoryanalysis}

%TOPIC: GENERAL STATEMENT
This section describes the collection of data, together with their validation and a detailed description of the modelled unit processes.
Like so, an explanation and accounting of reference flows are provided.
Essential parameters and variables are introduced.

\clearpage
\section{Single Loop}\label{sec:singleloop}

The modelling of a three-loop system required preparing the simulation of a single-cycle.
By commencing with a single loop, it became possible to identify those unit processes with higher emission contributions and allowed for a process-specific balance of the mass flow.
As previously mentioned, starting conditions for all product systems were identical.
However, differences become apparent right after the sorting process.

\section*{Mechanical}

%FIGURE: MECHANICAL SANKEY DIAGRAM
\begin{figure}[H]{\textwidth}
   \includegraphics[width=1\linewidth]{Figures/D7_SankeyMechanical}
   \caption{Disposal}
   \label{fig:7_sankey_mechanical}
\end{figure}

%TOPIC: SORTING
The first unit process (SORTED) was represented as a market activity from the ecoinvent database.
This process, being a market activity, meant that its emissions were those which started at the gate of any activity labelled as 'waste polyethene, for recycling, sorting.'
They regard not the long chain of previously inquired emissions from the manufacturing, but it accounts only for those concerning the most recent sorting activity.
Although no additional distances were added to the original market activity, it already represents the consumption mix of the product in the given geography (Europe without Switzerland).
Output was set at 1 kg.




\section*{Chemical}

%FIGURE: CHEMICAL SANKEY DIAGRAM
\begin{figure}[H]{\textwidth}
   \includegraphics[width=1\linewidth]{Figures/D7_SankeyChemical}
   \caption{Disposal}
   \label{fig:7_sankey_chemical}
\end{figure}
