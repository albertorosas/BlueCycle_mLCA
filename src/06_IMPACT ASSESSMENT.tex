\chapter{Impact Assessment}\label{ch:impactassesment}

%TOPIC: We doing IPCC 20a ✔
The impact assessment section of LCA aims to evaluate the prospective environmental impacts attributed to each and all of the elementary flows registered at the inventory phase.
Primarily, this calculation results from the association of the inventory data with a specific environmental impact category and category indicators.
This method would allow for the emergence of a reduced numerical listing which proclaims to encompass the environmental consequences attributed to the whole of the product system's processes.
Although there may be a great variety of impact assessment methodologies, each with a different logic behind the attribution of characterization, weighting and damage factors, this LCA exercise decided on utilizing IPCC GWP 20a.
This decision is related to an ambition for simplification and the prioritization granted by the authorities and the originator of this study.
Attached on \autoref{tab:IPCC} is a list of the impact factors utilized by the ecoinvent database over this specific assessment methodology.


\section{One Loop}\label{sec:singleresults}
\section*{Mechanical}\label{sec:mechanical2}

%FIGURE: MECHANICAL Bar GRaphs 1 LOOP
\begin{figure}[H]{\textwidth}
   \includegraphics[width=1\linewidth]{Figures/DU_M_1Loop_BarGraph}
   \caption{Top 5 contributors to impact category results. Single Cycle, Mechanical Recycling}
   \label{fig:U_M_1Loop_BarGraph}
\end{figure}

%TOPIC: The figure and the table ✔ %FIGURE: MECHANICAL BAR GRAPH %TABLE: RESULTING EMISSIONS, 1 LOOP
\autoref{fig:U_M_1Loop_BarGraph} and \autoref{tab:U_M_1Loop_TotalImpacts} depict the processes which contributed the most to the resulting emission on the MECHANICAL product system for a single loop.
As seen from these results, the most significant contributor to the emissions attributed to the mechanical recycling technology is its dependence on the input of fresh materials.
VIRGIN MATERIAL accounts for 43\% of the total emissions, significantly more than the INCINERATION of discarded materials itself.
Second on the list stands COLLECTION, attributed with 1,104 kg CO2 eq.  (30\%) out of the resulting 3,670 kg CO2 eq.
Both the figure and the table make visible the scale to which the crediting of avoided material reduces the total emission of the product system.
Reportedly, this amount adds up to 893 kg CO2 eq for processing 1,000 kg of DKR 350 stream.

%TABLE: RESULTING EMISSION, 1 LOOP MECHCNICAL
\begin{table}[H]
\caption{Impact analysis IPCC 2013 GMP 20a. Single Cycle, Mechanical Recycling}
\label{tab:U_M_1Loop_TotalImpacts}
\includegraphics[width=\textwidth]{Figures/DU_M_1Loop_TotalImpacts}
\end{table}


\section*{Chemical}\label{sec:chemical2}

%FIGURE: CHEMICAL Bar GRaphs 1 LOOP
\begin{figure}[H]{\textwidth}
   \includegraphics[width=1\linewidth]{Figures/DU_C_1Loop_BarGraph}
   \caption{Top 5 contributors to impact category results. Single Cycle, Chemical Recycling}
   \label{fig:U_C_1Loop_BarGraph}
\end{figure}

%TOPIC: All beauty in here. ✔ %FIGURE: CHEMCIAL Bar %TABLE: RESULTIng EMISION 1 LOOPS CHEMCIAL
Comparably to MECHANICAL, \autoref{fig:U_C_1Loop_BarGraph} and \autoref{tab:U_C_1Loop_TotalImpacts} report on the proportion of carbon equivalent released to the environment at each of the CHEMICAL system's processes.
These processes, even if more than those concerning MECHANICAL recycling,  amount to slightly less kg of CO2 eq.
Ostensibly, recycling 1,000 kg of DKR 350 through CHEMICAL (Pyrolysis) technology would emit 3,417 kg of CO2 eq. to the environment.
The LDPE PRODUCTION process emerged as the primary contributor with 39\% of the total system emissions.
Closely after came STEAM CRACKER with 1,114 kg CO2 eq. (32\%).
Contrary to the anticipation that STEAM CRACKER may account for the most significant emissions as it is a very energy-intensive process, it is worth remembering that the technological feature of reprocessing some of its outputs attributed to CHEMICAL recycling helps the technology to account for a reduced total.
\autoref{tab:U_C_1Loop_TotalImpacts} enlists all the emissions related to each process.
This table reports that the remaining credit amounted to 802.7 kg CO2 eq.(similar to MECHANICAL) even after subtracting those flows used inside each of the processes.

%TABLE: RESULTING EMISSION, 1 LOOP CHEMICAL
\begin{table}[H]
\caption{Impact analysis IPCC 2013 GMP 20a. Single Cycle, Chemical Recycling}
\label{tab:U_C_1Loop_TotalImpacts}
\includegraphics[width=\textwidth]{Figures/DU_C_1Loop_TotalImpacts}
\end{table}

\clearpage
\section{Three Loops}\label{sec:threeresults}
\section*{Mechanical and Chemical}\label{sec:mechanical4}

%TABLE: RESULTING EMISSION, 3 LOOPS MECHCNICAL
\begin{table}[H]
\caption{Impact analysis IPCC 2013 GMP 20a. Three Cycles, Mechanical Recycling}
\label{tab:mechanicalImpactAnalisis3F}
\includegraphics[width=\textwidth]{Figures/DU_M_3Loops_TotalImpacts}
\end{table}

%TOPIC: THIS ✔ %TABLE: RESULTING EMISSION MECHANICAL %TABLE: RESULTING EMISSION CHEMICAL
Expanding the simulation into the intended three-loop system accentuated the characteristical limitations of each technology.
\autoref{tab:mechanicalImpactAnalisis3F} and \autored{tab:chemcialImpactAnalisis3F} show the accumulated carbon-equivalent  emissions at each of the product system being compared.
When sequenced three times, MECHANICAL accumulated 9,422 kg of CO2 eq associated with processing 1,000 kg of DKR 350.
Meanwhile, CHEMICAL accomplished the same task with the collateral of 5,347 kg of CO2 eq.

%TOPIC: AND THAT ✔
As discussed, this is related to the crediting methodology that was utilized.
However, it also speaks of the inevitable circumstance of CHEMICAL being more efficient at reducing and repurposing the input product mass.
As MECHANICAL endures a percentual replenishment of virgin materials, each cycle remains to operate with extensive material amounts.
On the other hand, CHEMICAL ensures to continue reducing the output amount to a fraction that burdens less and less carbon emission at every iteration.
This effect is observable in their corresponding impact analyses.
Whereas VIRGIN MATERIAL will remain the highest emitting process across the three cycles at MECHANICAL, CHEMICAL's LDPE PRODUCTION process sets to release ever so less at every cycle.

%TABLE: RESULTING EMISSION, 3 LOOPS MECHCNICAL
\begin{table}[H]
\caption{Impact analysis IPCC 2013 GMP 20a. Three Cycles, Chemical Recycling}
\label{tab:chemcialImpactAnalisis3F}
\includegraphics[width=\textwidth]{Figures/DU_C_3Loops_TotalImpacts}
\end{table}

