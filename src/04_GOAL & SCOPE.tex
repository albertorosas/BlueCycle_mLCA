%\addcontentsline{toc}{chapter}{Goal \& Scope}
\chapter{Goal \& Scope Definition}\label{ch:goalandscope}

%TOPIC: In this chapter
The main objective of this phase was to institute the specifics and outline the technicalities of the selected product systems.
An exact approach to be followed is outlined.
In this chapter, the initial choices which determined the working plan are explained.
Details on the terms of geographical and temporal coverage, the products, reference flows, functions, functional units and system boundaries are presented.
Alongside, a description on the target audience, motives and the intended applications is included.


\clearpage
\section*{Goal and motivation}
%TOPIC: Motivation
This life cycle assessment, commissioned by Blue Cycle, was aimed to bring precision to the environmental impacts undergone by disparate dispositions of synthetic polymers.
Specifically, this study was directed to thermoplastic polyolefins in the form of mixed waste streams.
The main product systems in comparison were mechanical and chemical (pyrolitic) recycling.
As it was appropriate, this research aimed to first calculate the effects across these two particular processing routes and then have them compared to the alternative.
Although the analysis procured to include the most relevant processes across the several product life stages, additional attention was granted to the extraction, production and disposal stages.

%TOPIC: Current ways are crap
As previously mentioned, traditional multi-life cycle analyses have formerly leaned into reporting reuse and recycling practices almost indistinctly.
National legislation has suggested a simplified approach in which second-life material unambiguously replaces the usage of virgin material.
Consequently, emissions can be directly deducted on the scale of the amount that is being recycled.
However, even if simpler, this approach leaves out of sight differentiating attributes of any specific methodology.
Thereafter it effaces the, otherwise advantageous, characteristics of particular technological implementations.

%FIGURE: Full LCA with 2 lines
\begin{figure}[H]{\textwidth}
   \includegraphics[width=1\linewidth]{Figures/D3_BiColorMLCA}
   \caption{Chemical and mechanical recycling differentiation}
   \label{fig:3}
\end{figure}

%TOPIC: See, in the big picture, here is  the problem. %FIGURE: Full LCA with 2 lines.
\autoref{fig:3} shows the distinctive intermediate flows on both mechanical and chemical (pyrolysis) recycling.
As seen from the figure, material reflow is allocated into different stages.
Meanwhile second-life material retrieved from a mechanical operation would be reintroduced in the production phase, the pyrolitic product, reliant on additional processes, would substitute the use of new material in the extraction stage.
This of course is related to the specific properties prone to each of their outputs.
Yet, from a product system perspective, the overall releases and reference flows would aggregate ultimately differently even if the technologies are regarded as similar.

%FIGURE: ZOOM INTO EXTRACTION AND PRODUCTION
\begin{figure}[H]{\textwidth}
   \includegraphics[width=1\linewidth]{Figures/D4_ZoomIntoExtPro}
   \caption{Detailed view extraction and production life cycle stages}
   \label{fig:zoom}
\end{figure}

%TOPIC: About the zoom into extraction and production
\autoref{fig:zoom} indicates their differences across recycling methodologies more precisely.
As seen from the figure, the reintroduction of repurposed material is different in both stages and processes.
Meanwhile the product of mechanical recycling is typically reintroduced directly into the molding process, the liquid pyrolitic stream replaces the use of material at the cracking process, much closer to the raw material form.
Said differently, by being unique in their material properties these two methodologies will undoubtedly result in distinctive accounts of the energy inputs required to furnish each of the recycled products.
    %Further we will talk of how they are ultemlty different products but they perom the same functional unit
    %This process difference is inherate of the material properties each of the technologies can handle

%TOPIC: At disposal... Everything becomes different,
Accordingly to this assessment's ambition to recognize the environmental impacts across different recycling technologies, further distinctions are needed.
\autoref{fig:5Disposal} shows the disparate material flow ongoing on the current plastic disposal practices.
It is noticeable that intermediate product flows become distinctive subsequently to the sorting process.
It is also at this point where the individual product systems are set apart and undertake the remanufacturing function differently.
The circled numbers across \autoref{fig:zoom} and \autoref{fig:5Disposal} serve to identify these specific product streams and will, in due course, designate the material properties and mass balance across the life cycle modeling.

%FIGURE: ZOOM INTO DISPOSAL
\begin{figure}[H]{\textwidth}
   \includegraphics[width=1\linewidth]{Figures/D5_ZoomIntoDisposal}
   \caption{Disposal}
   \label{fig:5Disposal}
\end{figure}

%TOPIC: Moving on
Altogether, this initiative aspires on being able to evaluate and compare, under a normalized standard, the various environmental impacts of this, otherwise singular, product systems.

\clearpage
\section*{Functional Unit}
%TOPIC: What is a fucntional unit.
According to the ISO 14044:2006~\cite{ISO14044:2006}, the functional unit is understood as a reference unit which is used to quantified performance of the product system.
It is a measurable form of the product function, and its importance lays on providing with the ability to compare its fulfillment across products which perform this same function.
MIA/Vamil subsidy guidelines provided with a guideline onto multi-life functional unit definition.
In alignment to their instruction, the nominated functional unit reads:
\begin{displayquote}
    \textit{The processing of 1 tonne of DKR 350 plastic product into reusable virgin-grade material for at least 3 consecutive cycles.}
\end{displayquote}
However, it is believed, in continuation to the previously outlined discussion;
this functional unit further ignores some of the main attributes of chemical recycling technology.

%TOPIC: >>>>
\autoref{fig:6_twoproductsystems} shows the boundaries of product systems according to subsidy standards.
As it has been explained, it is not unexpected to resolve on higher emission associated to paralytic recycling when, even from first sight, several additional process and product stages are necessary.
For that reason, an alternative formulation is suggested:
\begin{displayquote}
    \textit{The delivery of (1,000) kJ of energy}
\end{displayquote}
Depicted in \autoref{fig:7}, an alternative comprehension of product system and boundaries is displayed.
This rephrasing of the product function is expected to more accurately manifest the sustainability attributes of the technologies in question.

%FIGURE: TWO PRODCUT SYSTEMS
\begin{figure}[H]{\textwidth}
   \includegraphics[width=1\linewidth]{Figures/D6_2ProductSystems}
   \caption{Disposal}
   \label{fig:6_twoproductsystems}
\end{figure}

%FIGURE: THRE PRODUCT SYSTEMS
\begin{figure}[H]{\textwidth}
   \includegraphics[width=1\linewidth]{Figures/D6_3ProductSystems}
   \caption{Disposal}
   \label{fig:7}
\end{figure}


% FIGURE to explain
\todo[inline]{\fontfamily{cmss}\selectfont
{"Although the rest of this file will continued to be guided by MIA/VAmil standards, it is suggested to the reader to keep these differentiation in mind as they both partake on the construction of a deaper understanding..."
}}
%TOPIC: BETO SIGUELE AQUI

% Secon figure relates to the disposal one and take in consideration teh other flows, not the traditional flows

\clearpage
\section*{Scope}
%TOPIC: Scope
    %PENDING: FULL PARAGRAPH

% About Scope
\todo[inline]{\fontfamily{cmss}\selectfont
{- Spatial Limitations: Netherlands and only neighboring countries. We are leaving outside of scope alternatives that may treat the plastic in other continents \\
 - Time limit: One year in preference, but we will see.\\
 - Technologies and health regulations are the ones imposed in europe.\\
 - Fuels, electricity and heat emission will be considered from current Dutch national energy mix.
}}





%%%%%%%%%%%%%%%%%%%%%%%%%%%%%%%%%%%%%%



%% DISPOSAL PHASE
%\todo[inline, caption={}]{\fontfamily{cmss}\selectfont
%{"A similar distinction occurs on the disposal phase..."
%The pending diagram will demonstrate further the distinctive processes under each of the technologies.
%The purpose of this schematics is to:
%
%    \begin{enumerate}
%    \item Show that despite being similar in reality both processes continue to be very different (this adding to the argument "They may both be apples, but we are talking green vs red apples in here\ldots)
%    \item Exhibit that ultimately all plastic materials will reach incineration.\\ This will be additionally supported by the mass balance that is being constructed by the circled numbers across the diagrams AND, most important, will lay the ground for the core of our argument.
%    Keeping with the apples, we are saying some like ("Actually, if you are trying to be healthier\ldots you should try to exercise (use liquid pyrolitic product as fuel) instead.;) )
%    \end{enumerate}
%
%}}
