%\addcontentsline{toc}{chapter}{Goal \& Scope}
\chapter{Goal \& Scope Definition}\label{ch:goalandscope}

%TOPIC: In this chapter
The main objective of this phase was to institute the specifics and outline the technicalities of the selected product systems.
An exact approach to be followed is outlined.
In this chapter, the initial choices which determined the working plan are explained.
Details on the terms of geographical and temporal coverage, the products, reference flows, functions, functional units and system boundaries are presented.
Alongside, a description on the target audience, motives and the intended applications is included.


\clearpage
\section*{Goal and motivation}
%TOPIC: Motivation
This life cycle assessment, commissioned by Blue Cycle, was aimed to bring precision on the environmental impacts undergone by disparate dispositions of synthetics polymers.
Specifically, this study was directed to thermoplastic polyolefins on the form of mixed waste streams.
The main product systems in comparison were mechanical and chemical (pyrolysis) recycling.
As it was appropriate, this research aimed to first calculate the effects across these two particular processing routes and then have them compared to the alternative.
Although the analysis procured to include the most relevant processes across the several product life stages, additional attention was granted to the extraction, production and disposal stages.

%FIGURE: Full LCA with 2 lines
\begin{figure}[H]{\textwidth}
   \includegraphics[width=1\linewidth]{Figures/D3_BiColorMLCA}
   \caption{Chemical and mechanical recycling differentiation}
   \label{fig:3}
\end{figure}

%TOPIC: Current ways are crap
As previously mentioned, traditional multi-life cycle analyses have formerly leaned into reporting reuse and recycling practices almost indistinctly.
National legislation has suggested a simplified approach in which second-life material unambiguously replaces the usage of virgin material.
Consequently, emissions can be directly deducted on the scale of the amount that is being recycled.
However, even if simpler, this approach leaves out of sight differentiating attributes of any specific methodology.
Thereafter it effaces the, otherwise advantageous, characteristics of particular technological implementations.

%TOPIC: See, in the big picture, here is  the problem. %FIGURE: The squares with 2 lines.
\autoref{fig:3} shows the distinctive intermediate flows on both mechanical and chemical (pyrolysis) recycling.
As seen from the figure, material reflow is allocated into utterly different stages.
Meanwhile second-life material retrieved from a mechanical operation would be reintroduce on the production phase, the pyrolitic product, reliant on additional processes, would substitute the use of new material in the extraction stage.
This of course is related to the specific properties prone to each of their outputs.
Yet, from a product system perspective, the overall releases and reference flows would aggregate ultimately differently even if the technologies are regarded as similar.

%FIGURE: ZOOM INTO EXTRACTION AND PRODUCTION
\begin{figure}[H]{\textwidth}
   \includegraphics[width=1\linewidth]{Figures/D4_ZoomIntoExtPro}
   \caption{Detailed view extraction and production life cycle stages}
   \label{fig:4}
\end{figure}
% FIGURE to explain
\todo[inline]{\fontfamily{cmss}\selectfont
{"Figure 1.2 identifies this difference more precisely "
}}

% DISPOSAL PHASE
\todo[inline, caption={}]{\fontfamily{cmss}\selectfont
{"A similar distinction occurs on the disposal phase..."
The pending diagram will demonstrate further the distinctive processes under each of the technologies.
The purpose of this schematics is to:

    \begin{enumerate}
    \item Show that despite being similar in reality both processes continue to be very different (this adding to the argument "They may both be apples, but we are talking green vs red apples in here...)
    \item Exhibit that ultimately all plastic materials will reach incineration.\\ This will be additionally supported by the mass balance that is being constructed by the circled numbers across the diagrams AND, most important, will lay the ground for the core of our argument. Keeping with the apples, we are saying some like ("Actually, if you are trying to be healthier... you should try to exercise (use liquid pyrolitic product as fuel) instead. ;) )
    \end{enumerate}

}}

%FIGURE: ZOOM INTO DISPOSAL
\begin{figure}[H]{\textwidth}
   \includegraphics[width=1\linewidth]{Figures/D5_ZoomIntoDisposal}
   \caption{Disposal}
   \label{fig:5}
\end{figure}

%FIGURE: TWO PRODCUT SYSTEMS
\begin{figure}[H]{\textwidth}
   \includegraphics[width=1\linewidth]{Figures/D6_2ProductSystems}
   \caption{Disposal}
   \label{fig:5}
\end{figure}

%FIGURE: THRE PRODUCT SYSTEMS
\begin{figure}[H]{\textwidth}
   \includegraphics[width=1\linewidth]{Figures/D6_3ProductSystems}
   \caption{Disposal}
   \label{fig:5}
\end{figure}



\section*{Scope}
%TOPIC: Scope
    %PENDING: FULL PARAGRAPH

% About Scope
\todo[inline]{\fontfamily{cmss}\selectfont
{- Spatial Limitations: Netherlands and only neighboring countries. We are leaving outside of scope alternatives that may treat the plastic in other continents \\
 - Time limit: One year in preference, but we will see.\\
 - Technologies and health regulations are the ones imposed in europe.\\
 - Fuels, electricity and heat emission will be considered from current Dutch national energy mix.
}}



