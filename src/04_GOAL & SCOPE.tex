%\addcontentsline{toc}{chapter}{Goal \& Scope}
\chapter{Goal \& Scope Definition}\label{ch:goalandscope}

%TOPIC: In this chapter ✔
The main objective of this phase was to institute the specifics and outline the technicalities of the selected product systems.
An exact approach to be followed is outlined.
This chapter explains the initial choices that determined the working plan.
Details on geographical and temporal coverage, the products, reference flows, functions, functional units and system boundaries are presented.
A description of the target audience, motives and intended applications is included.

\section*{Goal and motivation}

%TOPIC: Motivation ✔
This life cycle assessment, commissioned by Blue Cycle, aimed to bring precision to the environmental impacts of disparate dispositions of synthetic polymers.
Specifically, this study was directed to thermoplastic polyolefins in the form of mixed waste streams.
The central product systems in comparison were mechanical and chemical (pyrolitic) recycling.
As appropriate, this research aimed to first calculate the effects across these two particular processing routes and then compare them to the alternative.
Although the analysis was procured to include the most relevant processes across the several product life stages, additional attention was granted to the extraction, production and disposal stages.
%TOPIC: Current ways are crap ✔
As previously mentioned, traditional multi-life cycle analyses have formerly leaned into indistinctly reporting reuse and recycling practices.
National legislation has suggested a simplified approach in which second-life material unambiguously replaces the usage of virgin material.
Consequently, emissions can be directly deducted on the scale of the amount that is being recycled.
However, even if more straightforward, this approach leaves out of sight differentiating attributes of any specific methodology.
After that, it effaces the otherwise advantageous characteristics of particular technological implementations.

%FIGURE: Full LCA with 2 lines
\begin{figure}[H]{\textwidth}
   \includegraphics[width=1\linewidth]{Figures/D3_BiColorMLCA}
   \caption{Chemical and mechanical recycling differentiation}
   \label{fig:3}
\end{figure}

%TOPIC: See, in the big picture, here is  the problem. %FIGURE: Full LCA with 2 lines. ✔
\autoref{fig:3} shows the distinctive intermediate flows of mechanical and chemical (pyrolysis) recycling.
As seen from the figure, material reflow is allocated into different stages.
Meanwhile second-life material retrieved from a mechanical operation would be reintroduced in the production phase, the pyrolitic product, reliant on additional processes, would substitute the use of new material in the extraction stage.
This difference is related to the specific properties prone to each of their outputs.
Yet, from a product system perspective, the overall releases and reference flows would aggregate ultimately differently even if the technologies are regarded as similar.
%FIGURE: ZOOM INTO EXTRACTION AND PRODUCTION
\begin{figure}[H]{\textwidth}
   \includegraphics[width=1\linewidth]{Figures/D4_ZoomIntoExtPro}
   \caption{Detailed view extraction and production life cycle stages}
   \label{fig:zoom}
\end{figure}

%TOPIC: About the zoom into extraction and production ✔
\autoref{fig:zoom} indicates their differences across recycling methodologies more precisely.
As seen from the figure, the reintroduction of repurposed material is different in both stages and processes.
Meanwhile the product of mechanical recycling is typically reintroduced directly into the moulding process; the liquid pyrolitic stream replaces the use of the material at the cracking process, much closer to the raw material form.
Said differently, these two methodologies will undoubtedly result in distinctive accounts of the energy inputs required to furnish each recycled product by being unique in their material properties.
    %Further we will talk of how they are ultemlty different products but they perom the same functional unit
    %This process difference is inherate of the material properties each of the technologies can handle

%TOPIC: At disposal... Everything becomes different ✔
Accordingly to this assessment's ambition to recognize the environmental impacts across different recycling technologies, further distinctions are needed.
\autoref{fig:5Disposal} shows the disparate material flow ongoing on the current plastic disposal practices.
It is noticeable that intermediate product flows become distinctive subsequently to the sorting process.
At this point, the individual product systems are set apart and undertake the remanufacturing function differently.
%The circled numbers across \autoref{fig:zoom} and \autoref{fig:5Disposal} serve to identify these specific product streams and will, in due course, designate the material properties and mass balance across the life cycle modeling.

%FIGURE: ZOOM INTO DISPOSAL
\begin{figure}[h]{\textwidth}
   \includegraphics[width=1\linewidth]{Figures/D5_ZoomIntoDisposal}
   \caption{Disposal}
   \label{fig:5Disposal}
\end{figure}

%TOPIC: Moving on ✔
Altogether, this initiative aspires on being able to evaluate and compare, under a uniform standard, the various environmental impacts of this, otherwise singular, product systems.


\section*{Functional Unit}

%TOPIC: What is a fucntional unit. ✔
According to the ISO 14044:2006~\cite{ISO14044:2006}, the functional unit is understood as a reference unit used to quantify the product system's performance.
It is a measurable form of the product function, and its importance lies in the ability to compare its fulfilment across products which perform it.
MIA/Vamil subsidy guidelines provided a guideline for multi-life functional unit definition.
In alignment with their instruction, the nominated functional unit reads:
\begin{displayquote}
    \textit{The processing of 1 tonne of DKR 350 plastic product into reusable virgin-grade material for at least 3 consecutive cycles.}
\end{displayquote}
    %However, it is believed, in continuation to the previously outlined discussion;
    %this functional unit further ignores some of the main attributes of chemical recycling technology.

%TOPIC: Showing product units ✔
\autoref{fig:6_twoproductsystems} shows the boundaries of product systems according to subsidy standards.
As can be seen, although the comparing product system aggregates disparate amounts of unit processes, both are limited to the triple cycle requirement and have been set a boundary that excludes those processes not related to the recycling itself.
Emissions accounting began at the disposal phase and continued towards replacing inputs at extraction for chemical recycling and the production phase for mechanical recycling.

%FIGURE: TWO PRODCUT SYSTEMS
\begin{figure}[H]{\textwidth}
   \includegraphics[width=1\linewidth]{Figures/D6_2ProductSystems}
   \caption{Product system depiction including the three recycle cycles}
   \label{fig:6_twoproductsystems}
\end{figure}

\section*{Scope}
%TOPIC: Scope ✔
The assumptions made on this life cycle assessment were procured to replicate conditions from the Netherlands. Mostly, the data used was representative of Europe without Switzerland. Whenever not available, data sources were replaced with those which conditions resembled the most.
Fuels, electricity and the use of heat were all accounted for based on European standards.
Regarding the time frame, most sources were adjusted to a year spectrum.
However, considering the nature of the life span of plastics, mass flows were designated on the value of their availability and proximity.
This meant that the model accounted for the determined material quantity at the beginning of every cycle, even if it were not have been the same specific product.

%TOPIC: Assumptions ✔
Some assumptions were established and reconsidered under the opportunities and limitations brought by this computation system.
Most relevantly was the simplification of the DKR 350 stream into a 100\% LDPE composition.
Meanwhile it was initially ambitioned to replicate the fraction compositions of DKR 350 product flow across the system;
this method became impossible when consolidating product flows across data sources.
Secondly, it may have been the case that, although there was an exemplar process available in the database, the source was only accounted for as a foreign country.
Altogether these particulars were considered and scrutinised.
Neither of which were considered to disprove or refute the conclusions.

%TOPIC: Assumption, the one from transport ✔
Finally, most considerations for transport were excluded.
This decision came from early-in-the-process calculations where the sum of the transported distances at both of the product systems came out to be very similar (about 280 to 300 km)
Their aggregation would add up to almost identical transport emissions even if occurring at different life cycle stages.
In consequence, and aiming for a more straightforward modelling exercise, transport across processes was disregarded.
