\addcontentsline{toc}{section}{Introduction}
\section*{Introduction}\label{sec:introduction}

%% To be included
%\todo[inline]{\fontfamily{cmss}\selectfont
%{\textbf{To be included:}\\
%- About the different ways of recycling\\
%- About LCA a methodology\\
%}}

%TOPIC: Plastic is a problem...
    %PENDING: FULL PARAGRAPH

%TOPIC: About different ways of recycling but it is still very difficult
    %PENDING: FULL PARAGRAPH

%TOPIC: In the other hand, fuels sourcing and the current burning of trash is problematic.
    %PENDING: FULL PARAGRAPH

%TOPIC: Changing the fuel sources is necessary. and utilizing what other wise is lost molecules is imperative.
    %PENDING: FULL PARAGRAPH

%TOPIC: About LCA is a methodology, and works perfectly to outlline the pros and cons of different methods
The Life Cycle Analysis, hereafter LCA, is a methodology aimed to provide a quantitative evaluation across, otherwise divergent, design choices.
Frequently used in the development of new services or technologies, it allows computing the environmental impacts of disparate and several product systems.
Dating back to the 1960s, it has matured into a comprehensive decision-making tool that helps to establish a far-reaching perspective surrounding sustainability goals and ambitions.
Although predominantly related to the natural environment, it may be also tailored to include the accounting of social and economic effects while evaluating their plausible trade-offs. \cite{Hauschild2018, ISO14044:2006}

%TOPIC: The LCA Languaje %FIGURE: LINEAR LIFE CYCLE
\autoref{fig:1} shows a schematic interpretation of the LCA approach.
It serves to illustrate the succeeding product stages and aims to instantiate some of the definitions typically used in the methodology.
Conventionally, a life cycle study would rely on reckoning a collection of various input and output streams which take part in the elaboration of any product, or service,  since the extraction of materials in their natural environment up until the point it stops existing or it delivers its intended purpose (also known as a functional unit).
Inside the methodology jargon, this collection is referred to as inventory.
Inputs are subsequently distinguished into those in the form of materials, products or energy;
meanwhile, the specific outputs are referred to as releases, co-products or waste.
The most common product stages are simplified in \autoref{fig:1}.
Withal, inside these stages are considered the several unit processes which are responsible for transforming the inputs into outputs.
Depicted in black, in between the stages, are the reference flows.
These are respective to a measure of those outputs from any process or stage which continue advancing through the life cycle and are required for fulfilling the function of the functional unit.
Altogether, this collection of unit processes and product flows in tandem with their interactions is referred to as a product system.

%FIGURE: LINEAR LIFE CYCLE
\begin{figure}[H]
\centering
\begin{subfigure}[b]{\textwidth}
   \includegraphics[width=1\linewidth]{Figures/D1_ExplanatoryCycle}
   \caption{Traditional life cycle configuration}
   \label{fig:1}
\end{subfigure}

%FIGURE: MULTI LIFE CYCLE, BASIC LOOP
\begin{subfigure}[b]{\textwidth}
   \includegraphics[width=\linewidth]{Figures/D2_NormalMLCA}
   \caption{Multi-life cycle approach}
   \label{fig:2}
\end{subfigure}

\caption[]{Reference flow disparity between traditional and multi-life cycle analysis}
\end{figure}

%TOPIC: This is MLCA
In contrast from \autoref{fig:1}, \autoref{fig:2} shows the iterative utilization of otherwise discarded material or waste back into the product cycle.
Differently from the baseline understanding, this design approach seeks to lengthen the product's life cycle by reinstating reusable features and reducing the amount of material needed for its production.
Despite this practice could alternatively make usage of energy flows, the focus on material recovery is understood as multi-life cycle assessment.
\autoref{fig:2} serves to expose both the recycle feature and also the endmost disparity this change would on have on the quantitative analysis.

%TOPIC: About an mLCA
Although not always referred to as a multi-life cycle assessment, Suharinyanto et al.~\cite{Suhariyanto2017} demonstrated there has been an increased interest across literature on the development of life cycle analyses which include stages like remanufacturing, reusing and recycling inside their system boundaries.
This exhibits a transition beyond single-life cycle approaches and widens the connotation of cradle-to-grave definitions into a cradle-to-cradle understanding.
That is not to say the idea of an mLCA is new.
Terms like LCA for remanufacturing or LCA for recycling are common across literature ~\cite{Simon2016, Woolridge2006}
However, these were most commonly regarded towards products with low technological discontinuance, meaning they could easily be resold or be provided with a second life utilization (reused).
More complex aspects like a market lifetime and product replacement have usually been neglected ~\cite{Suhariyanto2017}.

