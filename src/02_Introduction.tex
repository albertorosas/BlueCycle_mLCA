\addcontentsline{toc}{section}{Summary}
\section*{Summary}\label{sec:summary}

%TOPIC: Todo empieza aqui ✔
In the context of an undergoing transition toward circularity and sustainable waste management, the Netherlands, through the Ministry of Infrastructure and Water Management, has deployed the Landelijk afvalbeheerplan (LAP3).
In this policy framework, the national government has included incentives to develop further the investment in environmentally friendly assets.
This document was made to serve on the petition for  Blue Cycle's prior investments to be included under Willekeurige afschrijving milieu-investeringen (Vamil) and Milieu-investeringsaftrek (MIA) subsidy schemes.
In line with that ambition, this undertaking may serve to justify and advocate for a deviation from incumbent waste hierarchical granting.
%A secondary but resembling examination was performed on... **{For the second stage I would like to consider the product as the hydrocarbon molecule}**

%TOPIC: So we did a m-LCA
%This hearing was tackled by means a Multi-life Cycle Assessment (mLCA).
%In contrast from traditional Life Cycle Assessment (LCA)+
    %PENDING: FULL PARAGRAPH

\newpage
\addcontentsline{toc}{section}{Introduction}
\section*{Introduction}\label{sec:introduction}

%TOPIC: Plastic is a problem...
    %PENDING: FULL PARAGRAPH

%TOPIC: About different ways of recycling but it is still very difficult
    %PENDING: FULL PARAGRAPH

%TOPIC: In the other hand, fuels sourcing and the current burning of trash is problematic.
    %PENDING: FULL PARAGRAPH

%TOPIC: Changing the fuel sources is necessary. and utilizing what other wise is lost molecules is imperative.
    %PENDING: FULL PARAGRAPH

%TOPIC: About LCA is a methodology, and works perfectly to outlline the pros and cons of different methods ✔
The Life Cycle Analysis, hereafter LCA, is a methodology that provides a quantitative evaluation across otherwise divergent design choices.
Frequently used in developing new services or technologies, it allows computing the environmental impacts of disparate and several production systems.
Dating back to the 1960s, it has matured into a comprehensive decision-making tool that helps to establish a far-reaching perspective surrounding sustainability goals and ambitions.
Although predominantly related to the natural environment, it may also be tailored to include the accounting of social and economic effects while evaluating their reasonable trade-offs. \cite{Hauschild2018, ISO14044:2006}

%TOPIC: The LCA Languaje %FIGURE: LINEAR LIFE CYCLE ✔
\autoref{fig:1} shows a schematic interpretation of the LCA approach.
It illustrates the succeeding product stages and aims to instantiate some of the definitions typically used in the methodology.
Conventionally, a life cycle study would rely on reckoning a collection of various input and output streams which take part in the elaboration of any product, or service,  since the extraction of materials in their natural environment up until the point it stops existing or it delivers its intended purpose (also known as a functional unit).
Inside the methodology jargon, this collection is referred to as inventory.
Inputs are subsequently distinguished into those in the form of materials, products or energy;
meanwhile, the specific outputs are releases, co-products or waste.
The most common product stages are simplified in \autoref{fig:1}.
Withal, these stages are considered the several unit processes responsible for transforming the inputs into outputs.
Depicted in black, in between the stages, are the reference flows.
These are respective to a measure of those outputs from any process or stage that continue advancing through the life cycle and are required to fulfil the functional unit.
This collection of unit processes and product flow in tandem with their interactions is referred to as a product system.

%FIGURE: LINEAR LIFE CYCLE
\begin{figure}[H]
\centering
\begin{subfigure}[b]{\textwidth}
   \includegraphics[width=1\linewidth]{Figures/D1_ExplanatoryCycle}
   \caption{Traditional life cycle configuration}
   \label{fig:1}
\end{subfigure}

%FIGURE: MULTI LIFE CYCLE, BASIC LOOP
\begin{subfigure}[b]{\textwidth}
   \includegraphics[width=\linewidth]{Figures/D2_NormalMLCA}
   \caption{Multi-life cycle approach}
   \label{fig:2}
\end{subfigure}
\caption[]{Reference flow disparity between traditional and multi-life cycle analysis}
\end{figure}

%TOPIC: This is MLCA %FIGURE: LINEAR LIFE CYCLE %FIGURE: MULTI LIFE CYCLE, BASIC LOOP ✔
In contrast to \autoref{fig:1}, \autoref{fig:2} shows the iterative utilisation of otherwise discarded material or waste back into the product cycle.
Differently from the baseline understanding, this design approach seeks to lengthen the product's life cycle by reinstating reusable features and reducing the amount of material needed for its production.
Despite this practice could alternatively make usage of energy flows, the focus on material recovery is understood as a multi-life cycle assessment.
\autoref{fig:2} exposes both the recycle feature and the endmost disparity this change would have on the quantitative analysis.

%TOPIC: About an mLCA ✔
Although not always referred to as a multi-life cycle assessment, Suharinyanto et al.~\cite{Suhariyanto2017} demonstrated there had been an increased interest across literature on the development of life cycle analyses which include stages like remanufacturing, reusing and recycling inside their system boundaries.
These findings exhibit a transition beyond single-life cycle approaches and widen the connotation of cradle-to-grave definitions into a cradle-to-cradle understanding.
That is not to say the idea of an mLCA is new.
Terms like LCA for remanufacturing or LCA for recycling are standard across literature ~\cite{Simon2016, Woolridge2006}
However, these were most commonly regarded as products with low technological discontinuance, meaning they could easily be resold or be provided with a second life utilisation (reused).
More complex aspects like a market lifetime and product replacement have usually been neglected ~\cite{Suhariyanto2017}.


\addcontentsline{toc}{section}{Procedures}
\section*{Procedure}\label{sec:procedure}

%TOPIC: Attributional LCA ✔
Overall, this research exercise remained along the character of an attributional LCA.
In difference from a consequential practice, this inquiry did not account for dynamic changes outside of systems boundaries in consequences of input variations.
Consequential descriptions as economic relationships, marginal production cost or the elasticity of supply and demands were not considered.
A confining, material-balanced approach was selected.

%TOPIC: About Mass Balance ✔
1,000 kg of DKR 350 material was modelled as an input across all the different product systems.
Both single and multi-loop calculations began with the acquisition of presorted recyclable material.
Nevertheless, the respecting mass balance across contrasted technologies became singular to their specific conditions.
This feature was paramount in the modelling as every subsequent unit process utilised only the output mass of its predecessor.

%TOPIC: The Databases used. ✔
The usage of specific inventory databases was also an enabling and limiting factor.
Three datasets were convened to realise this assessment: Ecoinvent v3.8, Product Environmental Footprints (PEF) and European Reference Life Cycle Database from the Joint Research Center v3.2.
These provided a comprehensive source of unit processes and elementary flow inventory but also confined the modelling exercise to how the author contemplated their boundaries.
At times, these may have overlapped and demanded a detailed verification to avoid double counting.
The latest version of OpenLCA software was used to fastly and reliably calculate the impacts across mentioned inventories.

%TOPIC: Assumptions ✔
Some assumptions were established and reconsidered under the opportunities and limitations brought by this computation system.
Most relevantly was the simplification of the DKR 350 stream into a 100\% LDPE composition.
Meanwhile it was initially ambitioned to replicate the fraction compositions of DKR 350 product flow across the system;
this method became impossible when consolidating product flows across data sources.
Secondly, it may have been the case that, although there was an exemplar process available in the database, the data was only accounted for as a foreign country.
Altogether these particulars were considered and scrutinised.
Neither of which were considered to disprove or refute the conclusions.

%TOPIC: TECHNICALLY NEITHER OF THE TECHNIQUES ARE THE SAME
    %PENDING: FULL PARAGRAPH
    %Not only is there a difference in the recicling and reusage understanding...
    %But even from the materials that are being used... After the sorting of the material... THE TREATMENT OF MONO FLWO AND PLY FLOWS IS DIFERENT FROM EACH TECHNNOLY
    %Now, because the comparison between these two methodologies comes as farfeched... It is better to look at a bigger picture... The one that comes from the ultimate usage of those carbon molecules.
