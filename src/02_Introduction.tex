\addcontentsline{toc}{section}{Introduction}
\section*{Introduction}\label{sec:introduction}

% To be included
\todo[inline]{\fontfamily{cmss}\selectfont
{\textbf{To be included:}\\
- About the different ways of recycling\\
- About LCA a methodology\\
}}

%TOPIC: Plastic is a problem...
    %PENDING: FULL PARAGRAPH

%TOPIC: About different ways of recycling but it is still very difficult
    %PENDING: FULL PARAGRAPH

%TOPIC: In the other hand, fuels sourcing and the current burning of trash is problematic.
    %PENDING: FULL PARAGRAPH

%TOPIC: Changing the fuel sources is necessary. and utilizing what other wise is lost molecules is imperative.
    %PENDING: FULL PARAGRAPH

%TOPIC: About LCA is a methodology, and works perfectly to outlline the pros and cons of different methods
    %PENDING: FULL PARAGRAPH

%FIGURE: LINEAR LIFE CYCLE
\begin{figure}[H]
\centering
\begin{subfigure}[b]{\textwidth}
   \includegraphics[width=1\linewidth]{Figures/Diagrams1}
   \caption{Traditional life cycle configuration}
   \label{fig:1}
\end{subfigure}

%FIGURE: MULTI LIFE CYCLE, BASIC LOOP
\begin{subfigure}[b]{\textwidth}
   \includegraphics[width=\linewidth]{Figures/Diagrams2}
   \caption{Multi-life cycle approach}
   \label{fig:2}
\end{subfigure}

\caption[]{Reference flow disparity between traditional and multi-life cycle analysis}
\end{figure}

%TOPIC: About an mLCA
Although not always referred as multi-life cycle assessment, Suharinyanto et al.~\cite{Suhariyanto2017} demonstrated there has been an increased interest across literature on the development of life cycle analyses which include stages like remanufacturing, reusing and recycling inside their system boundaries.
This, exhibits a transition beyond single-life cycle approaches and widens the connotation of cradle-to-grave definitions into a cradle-to-cradle understanding.
That is not to say the idea of a mLCA is new.
Terms like LCA for remanufacturing or LCA for recycling are common across literature ~\cite{Simon2016, Woolridge2006}
However, these were most commonly regarded to products with low technological discontinuance, meaning they could easily be resold or be provided with a second life utilization (reused).
More complex aspects like market lifetime and product replacement have usually been neglected ~\cite{Suhariyanto2017}.

