\chapter{Interpretation}\label{ch:interpretation}

%TABLE: FINAL Results
%% Please add the following required packages to your document preamble:
%% \usepackage{booktabs}
%% \usepackage{graphicx}
%\begin{table}[]
%\centering
%\caption{Final Results}
%\label{tab:my-tableloquas}
%\resizebox{\textwidth}{!}{%
%\begin{tabular}{@{}lcccc@{}}
%\toprule
%                    & \multicolumn{2}{c}{\textbf{Single Cycle}}                                                                                        & \multicolumn{2}{c}{\textbf{Three Cycles}}                                                                                        \\
%                    & \begin{tabular}[c]{@{}c@{}}Emissions\\ \[[kg CO2 eq{]}\end{tabular} & \begin{tabular}[c]{@{}c@{}}Material Output\\ {[}kg{]}\end{tabular} & \begin{tabular}[c]{@{}c@{}}Emissions\\ {[}kg CO2 eq{]}\end{tabular} & \begin{tabular}[c]{@{}c@{}}Material Output\\ {[}kg{]}\end{tabular} \\ \midrule
%\textbf{MECHANICAL} & 3,760                                                           & 850                                                            & 9,422                                                           & 610                                                            \\
%\textbf{CHEMICAL}   & 3,417                                                           & 400                                                            & 5,347                                                           & 65                                                             \\ \bottomrule
%\end{tabular}%
%}
%\end{table}

%TABLE: CORRECTED
% Please add the following required packages to your document preamble:
% \usepackage{booktabs}
% \usepackage{graphicx}
\begin{table}[]
\centering
\caption{Final Results}
\label{tab:my-tableloquas}
\resizebox{\textwidth}{!}{%
\begin{tabular}{@{}lcc@{}}
\toprule
                    & \textbf{Single Cycle}                                           & \textbf{Three Cycles}                                           \\
                    & \begin{tabular}[c]{@{}c@{}}Emissions\\ {[}kg CO2 eq{]}\end{tabular} & \begin{tabular}[c]{@{}c@{}}Emissions\\ {[}kg CO2 eq{]}\end{tabular} \\ \midrule
\textbf{MECHANICAL} & 3,760                                                           & 9,422                                                           \\
\textbf{CHEMICAL}   & 3,417                                                           & 5,347                                                           \\ \bottomrule
\end{tabular}%
}
\end{table}


%TOPIC: YA, se acabo
Even if inherently different, mechanical and chemical recycling technologies provide a kindred solution to the increasing torrent of plastic waste.
\autoref{tab:my-tableloquas} summarizes the abide material output after three consecutive recycling iterations of a 1,000 kg DKR 350 plastic stream and the equivalent carbon emissions attributes to each of these technologies.
As a matter of course, both of these technologies bear distinct advantages and disadvantages that may otherwise make them preferable under specific conditions.
However, as demonstrated in this multi-life cycle assessment, chemical recycling appears less impactful in both the leftover material residue and the inexorable carbon emissions released onto the environment.
