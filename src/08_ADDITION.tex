\chapter{Sensitivity Analysis}\label{ch:interpretation}

%TOPIC: INTRO TO SENTISITYVYT
The purpose of this section is to study the resulting alternatives to the initial modelling by considering alternative assumptions.
Two alternatives are presented.
ALT1. 100\% RECYCLED is defined by assuming that the MOULDING process from the MECHANICAL product system uses 100\% of recycled material.
This, also with the caveat that the only alternative after second-loop SORTING is INCINERATION.
There is not a third cycle.
All the collected products at the end of the first cycle are sorted and then directly burned.
This alternative is an example of some specific products like recycled-plastic trash bags.
Secondly, 100\% VIRGIN showcases the resulting emissions from not utilizing any recycling methodology.

\section*{ALT1 100\% RECYCLED}\label{sec:mechanical4}
%FIGURE: R_SANKEY
\begin{figure}[H]{\textwidth}
   \includegraphics[width=1\linewidth]{Figures/Z_100Recycled_SingleLoop_Sankey}
   \caption{Sankey Diagram ALT1 100\% RECYCLED}
   \label{fig:R_SANKEY}
\end{figure}

%TABLE: R_MASS FLOW
\begin{table}[H]
\caption{Intermediate Flow for ALT1 100\% Recycled}
\label{tab:R_MASSFLOW}
\includegraphics[width=\textwidth]{Figures/Z_100Recycled_SingleLoop_MassFlow}
\end{table}

%TOPIC: BLA
\autoref{fig:R_SANKEY} and \autoref{tab:R_MASSFLOW}
enlist and showcase the mass flow on this alternative scenario.
In contrast to the original MECHANICAL simulation, there is a significant reduction in the input mass.
The leftover product at the end of the first loop is only 350kg.
However, considering these results contemplate a subsequent and final INCINERATION, the released carbon emissions are higher, even after their corresponding credit.
\autoref{fig:R_BAR} and \autoref{tab:R_EMISSION} expose the increase in the total credits granted to the system and the change of the main contributor from being VIRGIN MATERIAL to now being the SORTED process.
1,134 kg of CO2 eq is attributed to this alternative.

%FIGURE: R_BAR
\begin{figure}[H]{\textwidth}
   \includegraphics[width=1\linewidth]{Figures/Z_100Recycled_SingleLoop_BarChart}
   \caption{Top 5 contributors to impact category results. Single Cycle, ALT1 100\% Recycled}
   \label{fig:R_BAR}
\end{figure}

%TABLE: R_EMISSION
\begin{table}[H]
\caption{Impact analysis IPCC 2013 GMP 20a. Single Cycle, ALT1 100\% Recycled}
\label{tab:R_EMISSION}
\includegraphics[width=\textwidth]{Figures/Z_100Recycled_SingleLoop_Emissions}
\end{table}


\section*{ALT2 100\% VIRGIN}\label{sec:mechanical4}
%FIGURE: V_SANKEY
\begin{figure}[H]{\textwidth}
   \includegraphics[width=1\linewidth]{Figures/Z_100Virgin_SingleLoops_SAnkey}
   \caption{Sankey Diagram ALT2 100\% Virgin}
   \label{fig:V_SANKEY}
\end{figure}

%TABLE: V_MASSFLOW
\begin{table}[H]
\caption{Intermediate Flows for ALT2 100\% Virgin}
\label{tab:V_MASSFLOW}
\includegraphics[width=\textwidth]{Figures/Z_100Virgin_SingleLoops_MassFlow}
\end{table}

%FIGURE: V_BAR
\begin{figure}[H]{\textwidth}
   \includegraphics[width=1\linewidth]{Figures/Z_100Virgin_SIngleLoops_BarChart}
   \caption{Top 5 contributors to impact category results. Single Cycle, ALT2 100\% Virgin}
   \label{fig:V_BAR}
\end{figure}

%TABLE: V_EMISSIONS
\begin{table}[H]
\caption{Impact analysis IPCC 2013 GMP 20a. Single Cycle, ALT2 100\% Virgin}
\label{tab:V_EMISSION}
\includegraphics[width=\textwidth]{Figures/Z_100Virgin_SingleLoop_Emissions}
\end{table}

%TOPIC: AVOIDANCE IS RECOMENDED :)
ALT2 100\%VIRGIN considered an additional possibility, the case for non-participating in any recycling practice.
\autoref{fig:V_SANKEY} and \autoref{tab:V_MASSFLOW} detail in a much simpler product sytem.
No reduction in mass is considered.
Interestingly, although unsurprising, the total carbon emission related to a single cycle with 100\% virgin material is the highest across all scenarios.
5,243 kg of CO2 eq is attributed to the manufacturing, MOULDING and subsequent COLLECTION of 1,000 kg of LDPE plastic product.
\autoref{tab:my-final} shows these resulting values compared to the original considerations on this LCA.
Reinstated in \autoref{fig:V_BAR} and \autoref{tab:V_EMISSION}, the enduring utilization of virgin material in the production of LDPE plastic products is clear to have the most significant carbon emission and the utmost impact on the environment.
Its avoidance is recommended.

%TABLE: NEW FINAL RESULTS
% Please add the following required packages to your document preamble:
% \usepackage{graphicx}
\begin{table}[H]
\centering
\caption{Final Results}
\label{tab:my-final}
%resizebox{\textwidth}{!}{%
\begin{tabular}{lcc}
\hline
                    & \textbf{Single Cycle}                                           & \textbf{Three Cycles}                                           \\
                    & \begin{tabular}[c]{@{}c@{}}Emissions\\ {[}kg CO2 eq{]}\end{tabular} & \begin{tabular}[c]{@{}c@{}}Emissions\\ {[}kg CO2 eq{]}\end{tabular} \\ \hline
\textbf{MECHANICAL} & 3,760                                                           & 9,422                                                           \\
\textbf{CHEMICAL}   & 3,417                                                           & 5,347                                                           \\ \hline
ALT1 100\% RECYCLED & 1,134                                                           & \multicolumn{1}{l}{}                                            \\
ALT2 100\%VIRGIN    & 5,243                                                           & \multicolumn{1}{l}{}
\end{tabular}%
%}
\end{table}
